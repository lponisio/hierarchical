\documentclass[12pt]{article} 
\usepackage[pdftex]{graphicx}
\usepackage{natbib} 
\usepackage{color}
\usepackage{amsmath} 
\usepackage{amssymb} 
\usepackage{verbatim}
\usepackage{mathpazo} 
\usepackage{setspace}
\usepackage{multirow}
\usepackage{fullpage}
\usepackage{lscape}
\usepackage{fancyhdr}
\usepackage[normalem]{ulem} 
\usepackage{hyperref}
\usepackage[parfill]{parskip}
\usepackage{graphicx}
\usepackage{textcomp}
\usepackage{xr}

\hypersetup{colorlinks=true, linkcolor=black, citecolor=black}
\RequirePackage{lineno}

\newcommand{\flagged}[1] {
  \textcolor{blue}{#1}
}

\def\title{Something occupancy models}
\def\author{Lauren C.\ Ponisio$^{1,2}$, Nicholas Michaud$^1$, Perry de
  Valpine$^1$}

\def\runninghead{Occupancy model efficiency}
\def\keywords{NIMBLE, Markov chain Monte Carlo, latent states, block
  sampling, dynamic occupancy, mutli species occupancy, spatial
  occupancy, JAGS }

\def\extras{
  \begin{itemize}
  \item Submitted as a Standard paper
  \item Abstract word count: 
  \item Main text word count: 
  \item Number of references: 
  \item Number of figures:
  \end{itemize}
}

\def\affiliation{
  \begin{enumerate}
  \item Department of Environmental Science, Policy, and Management\\
    University of California, Berkeley\\
    130 Mulford Hall\\
    Berkeley, California, USA\\
    94720\\
  \item Department of Entomology\\
    University of California, Riverside\\
    417 Entomology Bldg.\\
    Riverside, California, USA\\
    92521\\
  \end{enumerate}
}

\newcommand{\mstitlepage}{
  \paragraph{Running head:} \textsc{\runninghead}
  % \parindent=0pt
  \begin{center}%
    {\LARGE \title \par}%
    \vskip 3em%
    {\large
      \lineskip .75em%
      \begin{tabular}[t]{c}%
        \author
      \end{tabular}\par}%
    \vskip 1.5em%
  \end{center}\par
  \affiliation
}
\clearpage

\begin{document}

\mstitlepage
\doublespacing
\linenumbers
\clearpage

\begin{abstract}  
  \begin{enumerate}
  \item occupancy models are everywhere, but model fitting and
    assessment are extremely computationally intensive
  \item because models are so computationally intensive, users often
    forgo model assessment (determining if a model provides an
    adequate fit to a particular dataset) because if involves
    simulating from and refitting the model many times.
  \item Using the NIMBLE package for R, we develop combined
    computational approaches including user-defined and automatic
    blocking of parameters for MCMC, filtering over latent states, and
    customized MCMC samplers for specific parameters to improve
    efficiency. We test these approaches using three representative
    occupancy models of varying levels of complexity including a
    single species model with spatial auto-correlation, a single
    species dynamic (multi-season) model, and a multi-species
    model. We also develop and implement methods for calculating
    calibrated predictive posterior $p$-values to assess model fit
    within the open source modeling software, NIMBLE.
  \item These computation approaches lead to an improvement in MCMC
    sampling efficiency over, particularly with models including
    random effects.
  \item Ours results highlight the need for more customizable
    approaches to MCMC to fit and assess hierarchical models in order
    to ensure occupancy models are accessible to practitioners. By
    implementing MCMC procedures and model assessment techniques in
    open source software, we have made progress toward this aim.
  \item \textit{Implications:}
  \end{enumerate}
\end{abstract}

\keywords

\clearpage

\section*{Introduction}
\label{sec:introduction}

Estimating the proportion of sites occupied by a species is common
challenge for many sub-disciplines ecology and evolution including
meta-population, endangered species and invasion biology. Greater
acceptance of the biases introduced by imperfect detection has lead to
the development and proliferation of occupancy models --- models where
the occurrence of a species at a site as a latent state layered
underneath a detection process \citep[e.g.,][]{mackenzie-2006,
  royle-2007-1813}. Now only a little over a decade after occupancy
models were introduced to ecology, they are being used to model the
occurrence of everything from bees \citep{mgonigle-2015-x} to tigers
\citep{hines2010tigers} in an endless variety of complexity.

Occupancy models are part of a larger class of models known as Hidden
Markov Models. For discrete Hidden Markov Models like occupancy models
where a species is either present or absent from a site, likelihood
calculation involves summing over the distribution of latent
states. Because estimating the effect of explanatory variables on site
occupancy or shared variation of in occupancy across species is often
of greatest interest to ecologists
\citep[e.g.,][]{iknayan2014detecting}, the Hidden Markov Models are
embedded within a hierarchical model. In such cases, practitioners
generally rely on Markov chain Monte Carlo (MCMC) to perform a
Bayesian analysis. Standard MCMC software will including the latent
state variables in MCMC sampling \citep[e.g.,][]{plummer-2003-jags,
  winbugs, openbugs}. Such models are computationally intensive, and
large models requiring hundreds or thousands of dimensions which
require MCMC can be intractable.

In addition, fitting these models is such a challenge that users often
forgo adding any additional computation to asses model fit. A common
idea behind evaluating whether a model provides an adequate fit to a
particular dataset is that if data is simulated from the model, the
simulated data should resemble the observed data. This is the basis of
posterior predictive $p$-values, which compare the distribution of
summary statistics calculated from simulated datasets to the observed
statistic. Posterior predictive $p$-values alone, however, often fail
to reject poor-fitting models \citep{bayarri-berger-00,
  robins-etal-00, hjort-etal-06}. Methods for correcting posterior
predictive $p$-values for better performance have been proposed
\citep[e.g., calibrated posterior predictive $p$-values,
][]{hjort-etal-06}, but refitting the model via MCMC iterativly. Given
that with occupancy models fitting the model just once can be a time
consuming task, efficient methods for MCMC are necessary to ensure
methods for assessment are feasible for these models.

Beyond assessing the fit of a model, choosing between models is one of
the most widely used applications of statistics by
practitioners. Though many theoretically sound methods for Bayesian
model selection such as cross-validation have been developed
\citep{hooten2014guide}, they, like model assessment, are
computationally intensive --- particularly for hierarchical models
like occupancy models. A typical need for model selection arises when
a practitioner is choosing whether to include a specific layer of
hierarchy (i.e., random effect). This is often the case with so called
``multi-species'' occupancy models, where the occupancy of many
species is estimated simultaneously in a model with a random effect of
species \citep[reviewed in, ][]{iknayan2014detecting}. Ecologists are
often interested in whether there is some variability in the response
of species to an explanatory variable such that a random effect of
species accounts for that variability, or whether a fixed effect of
that variable fits adequately
\citep{pacifici2014guidelines}. Currently, the Deviance Information
Criteria (DIC), originally derived to mimic AIC for Bayesian,
non-hierarchical models, is now commonly used by scientists to
evaluate hierarchical models. Though the limitations of DIC for
hierarchical model selection are widely recognized by statisticians
\citep{celeux2006deviance, hooten2014guide}, because it is built into
open-source software such as WinBUGS \citep{winbugs}, it is
uncritically used by practitioners. Readily available and
theoretically sound alternative methods are thus critically needed.


% Filtering over latent states to calculate model likelihoods inorder to
% limit MCMC sampling to top-level parameters dyanmic blocking or
% parameters \citep{turek2016efficient}




\section*{Materials \& Methods}
\label{sec:methods}
\subsection*{Computational approaches}

\subsubsection*{Single species, single season occupancy model with
  spatial auto-correlation}
The first model is a single species, single season occupancy model
accounting for spatial auto-correlation. We let $z_{i}$ denote the
true occupancy of a species at site $i$.  We then let $x_{i,j}$
indicate whether we detected ($x_{i,j}=1$) or did not detect
($x_{i,j}=0$) that species in the $j^{\mathrm{th}}$ visit to site $i$.
We assumed that occupancy at the $i^{\mathrm{th}}$ site is a Bernoulli
random variable $z_{i} \sim \mathrm{Bern}(\psi_{i})$ with probability
$\psi_{i}$.  We included the effect of an arbitrary covariate (e.g.,
elevation) on site occupancy. To model the spatial auto-correlation in
occupancy between sites, we assume the co-variance between sites $Y_i$
and $Y_j$ is a function of distance between $p_i$ and $p_j$. We
computed the probability of occupancy at site $i$
%
\begin{equation}
  \begin{split}
    \label{eq:spatial}
    \mathrm{logit}(\psi_{i}) &=
    \alpha + \beta*elevation_i + \rho_i\hspace{0.2em}\\
    \rho_i &\sim MVN(0, {Cov}(Y_i,Y_j)) \hspace{0.2em}\\
    {Cov} (Y_i,Y_j) &= \sigma^2exp^{(-\lambda\|p_i-p_j\|)} \hspace{0.2em}.\\
  \end{split}
\end{equation}
%

Where $\lambda$ is the exponential decay constant and $\sigma^2$ is
SOMETHING....

To improve efficiency of this model...

\subsubsection*{Single species, multi season (dynamic) occupancy
  model}
\label{sec:ssms}

The second model is a relatively simple single species occupancy model
over multiple seasons \citep{Royle2007}. We let $z_{i,t}$ denote the
true occupancy of a species in year $t$ at site $i$.  We then let
$x_{i,t,j}$ indicate whether we detected ($x_{i,t,j}=1$) or did not
detect ($x_{i,t,j}=0$) that species in the $j^{\mathrm{th}}$ visit to
site $i$ in year $t$.  We assumed that occupancy at the
$i^{\mathrm{th}}$ site in the $t^{\mathrm{th}}$ year is a Bernoulli
random variable $z_{i,t} \sim \mathrm{Bern}(\psi_{i,t})$ with
probability $\psi_{i,t}$.

Letting $\phi_{i,t}$ denote the probability the species persists at
site $i$ from years $t$ to $t+1$ (provided it was present at site $j$
in year $t$, $z_{i,t}=1$) and $\gamma_{i,t}$ denote the
probability that site $i$ is colonized in year $t+1$
(provided it was not present at site $i$ in year $t$, $z_{i,t}=0$),
we then computed the probability of occupancy  at site
$i$ in subsequent years as
%
\begin{equation}
  \label{eq:occupancy2}
  \psi_{i,t+1} =
  \phi_{i,t} * z_{i,t} + \gamma_{i,t} * (1-z_{i,t})\hspace{0.2em}.
\end{equation}
%

First, to improve efficiency, filter over latent states
to calculate model likelihoods in order to limit MCMC sampling to
top-level parameters. We then use two computational approaches to
improve the efficiency of this model 1) dynamic blocking of the
parameters \citep{turek2016efficient}, and 2) a custom MCMC
specification where a slice sampler is used for all parameters.

\subsubsection*{Multi species, single season occupancy model}
\label{sec:msss}

The last model is from \citep{zipkin2010multi}. It is a multi-species,
single season occupancy model examining the effect of wildlife
management and habitat characteristics on bird communities
\citep{zipkin2010multi}. The species-specific coefficients for the
effect of basal tree area, understory foliage and deer management
where bound together by a common distribution with an estimated
variance. The model is similar to Eq.~\ref{eq:spatial}, except each for
species $k$, we let $z_{i,k}$ denote its true occupancy state at site
$i$.
 
To improve the efficiency of this model, we first filtered over latent
states to calculate model likelihoods in order to limit MCMC sampling
to top-level parameters. We also vectorized all calculations that
would have require for loops in BUGS or JAGS. We then tried two
approaches to speed sampling of the top-level parameters 1) dynamic
blocking of the parameters \citep{turek2016efficient}, and 2) a custom
blocking scheme were the parameters of each species are blocked
together.


\subsection*{Model assessment}
\label{sec:assess}

We implemented a procedure to calculate calibrated posterior
predictive $p$-values \citep{hjort-etal-06}. After the parameters have
been fit to the model, a sample of the posterior is used to simulate
data from the model. A discrepancy measure, which we chose to be the
model likelihood, is then calculated, and the posterior $p$-value is
the number of simulated $p$-values that fall below the observed. To
``calibrate'' the distribution of posterior $p$-values, the MCMC is
rerun on the simulated data to refit the model. 

\subsection*{Model selection}
\label{sec:select}

Cross-validation is one of the most fudamental procedures in model
selection, but, because it requires iternativly re-fitting the model,
is computationally intensive \citep{hooten2014guide}. In cross
validation, we exclude a subset of the data ($y_k$) from model
fitting, then use the fitted model to predict $y_k$. The preidtion
error is summarized by coparing the simulated $y_k$ to the true $y_k$.

In the multi species occupancy models (Section \ref{sec:msss}),
practitioners are often interested in determining whether a model
including a species random effect for explanatory variables is a
better fit than a model without the random effect. We implemented a
cross-validation procudure for this model where the detection data for
species is left out, the model refitted, and the fitted model used to
predict the occurence of that species. The predictive error of the
model included a random effect of species is then compared to a model
where no species random effects were included.


\section*{Results}
\label{sec:results}

\section*{Discussion}
\label{sec:discussion}

\section*{Acknowledgments}
\label{sec:acknowledge}

\bibliographystyle{mee}

\bibliography{refs}

% \begin{figure}
%   \centering
%   \includegraphics[width=.8\textwidth]{../analysis/changePoint/plotting/networks.pdf}
%   \caption{Assembling hedgerow networks had more changing points
%     (vertical red lines) than non-assembling hedgerows and weedy field
%     margins (a representative sample of non-assembling sites are
%     depicted here). In each network, plants and pollinators are
%     represented by green and yellow circles, respectively, weighted by
%     their degree. Each species has a consistent position in the
%     network across years. In the assembling hedgerows, colored squares
%     in the corner of each network represent the year of
%     assembly. Before hedgerows are installed (when they are still
%     field margins) the year of assembly is zero.}
%   \label{fig:changePoints}
% \end{figure}
% \clearpage

\end{document}

%%% Local Variables:
%%% mode: latex
%%% TeX-PDF-mode: t
%%% End:
