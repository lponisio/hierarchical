\documentclass[12pt]{article} 
\usepackage[pdftex]{graphicx}
\usepackage{natbib} 
\usepackage{color}
\usepackage{amsmath} 
\usepackage{amssymb} 
\usepackage{verbatim}
\usepackage{mathpazo} 
\usepackage{setspace}
\usepackage{multirow}
\usepackage{fullpage}
\usepackage{lscape}
\usepackage{fancyhdr}
\usepackage[normalem]{ulem} 
\usepackage{hyperref}
\usepackage[parfill]{parskip}
\usepackage{graphicx}
\usepackage{textcomp}
\usepackage{xr}

\hypersetup{colorlinks=true, linkcolor=black, citecolor=black}
\RequirePackage{lineno}

\newcommand{\flagged}[1] {
  \textcolor{blue}{#1}
}

\def\title{Something occupancy models}
\def\author{Lauren C.\ Ponisio$^{1,2}$, Nicholas Michaud$^1$, Perry de
  Valpine$^1$}

\def\runninghead{Occupancy model efficiency}
\def\keywords{NIMBLE, Markov chain Monte Carlo, latent states, block
  sampling, dynamic occupancy, mutli species occupancy, spatial
  occupancy, JAGS }

\def\extras{
  \begin{itemize}
  \item Submitted as a Standard paper
  \item Abstract word count: 
  \item Main text word count: 
  \item Number of references: 
  \item Number of figures:
  \end{itemize}
}

\def\affiliation{
  \begin{enumerate}
  \item Department of Environmental Science, Policy, and Management\\
    University of California, Berkeley\\
    130 Mulford Hall\\
    Berkeley, California, USA\\
    94720\\
  \item Department of Entomology\\
    University of California, Riverside\\
    417 Entomology Bldg.\\
    Riverside, California, USA\\
    92521\\
  \end{enumerate}
}

\newcommand{\mstitlepage}{
  \paragraph{Running head:} \textsc{\runninghead}
  % \parindent=0pt
  \begin{center}%
    {\LARGE \title \par}%
    \vskip 3em%
    {\large
      \lineskip .75em%
      \begin{tabular}[t]{c}%
        \author
      \end{tabular}\par}%
    \vskip 1.5em%
  \end{center}\par
  \affiliation
}
\clearpage

\begin{document}

\mstitlepage
\doublespacing
\linenumbers
\clearpage

\begin{abstract}  
  \begin{enumerate}
  \item occupancy models are everywhere, but model fitting and
    assessment are extremely computationally intensive
  \item because models are so computationally intensive, users often
    forgo model assessment (determining if a model provides an
    adequate fit to a particular dataset) because if involves
    simulating from and refitting the model many times.
  \item Using the NIMBLE package for R, we develop combined
    computational approaches including user-defined and automatic
    blocking of parameters for MCMC, filtering over latent states, and
    customized MCMC samplers for specific parameters to improve
    efficiency. We test these approaches using three representative
    occupancy models of varying levels of complexity including a
    single species model with spatial auto-correlation, a single
    species dynamic (multi-season) model, and a multi-species
    model. We also develop and implement methods for calculating
    calibrated predistive posterior $p$-values to assess model fit
    within the open source modeling software, NIMBLE.
  \item These computation approaches lead to an improvement in MCMC
    sampling efficiency over, particularly with models including
    random effects.
  \item Ours results highlight the need for more customizable
    approaches to MCMC to fit and assess hierarchical models in order
    to ensure occupancy models are accessible to practitioners. By
    implementing MCMC procedures and model assessment techniques in
    open source software, we have made progress toward this aim.
  \item \textit{Implications:}
  \end{enumerate}
\end{abstract}

\keywords

\clearpage

\section*{Introduction}
\label{sec:introduction}

Estimating the proportion of sites occupied by a species is common
challenge for many sub-disciplines ecology and evolution including
meta-population, endangered species and invasion biology. Greater
acceptance of the baises introduced by imperfect detection has lead to
the development and proliferation of occupancy models --- models where
the occurrence of a species at a site as a latent state layered
underneath a detection process \citep[e.g.,][]{mackenzie-2006,
  royle-2007-1813}. Now only a little over a decade after occupancy
models were introduced to ecology, they are being used to model the
occurrence of everything from bees \citep{mgonigle-2015-x} to tigers
\citep{hines2010tigers} in an endless variety of complexity.

Occupancy models are part of a larger class of models known as Hidden
Markov Models. For discrete Hidden Markov Models like occupancy models
where a species is either present or absent from a site, likelihood
calculation involves summing over the distribution of latent
states. Because estimating the effect of explanatory variables on site
occupancy or shared variation of in occupancy across species is often
of greatest interest to ecologists
\citep[e.g.,][]{iknayan2014detecting}, the Hidden Markov Models are
embedded within a hierarchical model. In such cases, practitioners
generally rely on Markov chain Monte Carlo (MCMC) to perform a
Bayesian analysis. Standard MCMC software will including the latent
state varaibles in MCMC sampling \citep[e.g.,][]{plummer-2003-jags,
  winbugs, openbugs}. Such models are computationally intensive, and
large models requiring hundreds or thousands of dimensions which
require MCMC can be intractable. Filtering over latent states to
calculate model likelihoods inorder to limit MCMC sampling to
top-level parameters \citep{turek2016efficient}

In addition, fitting these models is such a challenge that users often
forgo adding any additional computation to asses model fit.  A common
idea behind evaluating whether a model provides an adequate fit to a
particular dataset is that if data is simulated from the model, the
simulated data should resemble the observed data. This is the basis of
posterior predictive $P$-values, which compare the distribution of
summary statistics calculated from simulated datasets to the observed
statistic. Posterior predictive $P$-values alone, however, often fail
to reject poor-fitting models \citep{bayarri-berger-00,
  robins-etal-00, hjort-etal-06}. Methods for correcting posterior
predictive $P$-values for better performance have been proposed
\citep[e.g., calibrated posterior predictive $P$-values,
][]{hjort-etal-06}, but refitting the model via MCMC iterativly.

To ensure occupancy models are accessible to practitioners, more
efficient methods for fitting and assessing these models are
necessary.



\section*{Materials \& Methods}

A common idea behind evaluating whether a model provides an adequate
fit to a particular dataset is that if data is simulated from the
model, the simulated data should resemble the observed data. This is
the basis of posterior predictive $P$-values, which compare the
distribution of summary statistics calculated from simulated datasets
to the observed statistic. Posterior predictive $P$-values alone,
however, often fail to reject poor-fitting models
\citep{bayarri-berger-00, robins-etal-00, hjort-etal-06}. Methods for
correcting posterior predictive $P$-values for better performance have
been proposed, but previously were too computationally intensive to be
feasible \citep[e.g., calibrated posterior predictive $P$-values,
][]{hjort-etal-06}.

\label{sec:methods}


\section*{Results}
\label{sec:results}

\section*{Discussion}
\label{sec:discussion}

\section*{Acknowledgments}
\label{sec:acknowledge}

\bibliographystyle{mee}

\bibliography{refs}

% \begin{figure}
%   \centering
%   \includegraphics[width=.8\textwidth]{../analysis/changePoint/plotting/networks.pdf}
%   \caption{Assembling hedgerow networks had more changing points
%     (vertical red lines) than non-assembling hedgerows and weedy field
%     margins (a representative sample of non-assembling sites are
%     depicted here). In each network, plants and pollinators are
%     represented by green and yellow circles, respectively, weighted by
%     their degree. Each species has a consistent position in the
%     network across years. In the assembling hedgerows, colored squares
%     in the corner of each network represent the year of
%     assembly. Before hedgerows are installed (when they are still
%     field margins) the year of assembly is zero.}
%   \label{fig:changePoints}
% \end{figure}
% \clearpage

\end{document}

%%% Local Variables:
%%% mode: latex
%%% TeX-PDF-mode: t
%%% End:
