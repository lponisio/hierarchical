\documentclass[12pt]{article} 
\usepackage[pdftex]{graphicx}
\usepackage{natbib} 
\usepackage{color}
\usepackage{amsmath} 
\usepackage{amssymb} 
\usepackage{verbatim}
\usepackage{mathpazo} 
\usepackage{setspace}
\usepackage{multirow}
\usepackage{fullpage}
\usepackage{lscape}
\usepackage{fancyhdr}
\usepackage[normalem]{ulem} 
\usepackage{hyperref}
\usepackage[parfill]{parskip}
\usepackage{graphicx}
\usepackage{textcomp}
\usepackage{xr}

\hypersetup{colorlinks=true, linkcolor=black, citecolor=black}
\RequirePackage{lineno}

\newcommand{\flagged}[1] {
  \textcolor{blue}{#1}
}

\def\title{Increasing the tractability of occupancy models}
\def\author{Lauren C.\ Ponisio$^{1,2}$, Perry de Valpine$^1$}

\def\runninghead{Occupancy model efficiency}
\def\keywords{NIMBLE, Markov chain Monte Carlo, latent states, block
  sampling, dynamic occupancy, mutli species occupancy, spatial
  occupancy, JAGS }

\def\extras{
  \begin{itemize}
  \item Submitted as a Standard paper
  \item Abstract word count: 
  \item Main text word count: 
  \item Number of references: 
  \item Number of figures:
  \end{itemize}
}

\def\affiliation{
  \begin{enumerate}
  \item Department of Environmental Science, Policy, and Management\\
    University of California, Berkeley\\
    130 Mulford Hall\\
    Berkeley, California, USA\\
    94720\\
  \item Department of Entomology\\
    University of California, Riverside\\
    417 Entomology Bldg.\\
    Riverside, California, USA\\
    92521\\
  \end{enumerate}
}

\newcommand{\mstitlepage}{
  \paragraph{Running head:} \textsc{\runninghead}
  % \parindent=0pt
  \begin{center}%
    {\LARGE \title \par}%
    \vskip 3em%
    {\large
      \lineskip .75em%
      \begin{tabular}[t]{c}%
        \author
      \end{tabular}\par}%
    \vskip 1.5em%
  \end{center}\par
  \affiliation
}
\clearpage

\begin{document}

\mstitlepage
\doublespacing
\linenumbers
\clearpage

\begin{abstract}  
place holder
\end{abstract}

\keywords

\clearpage

\section*{Introduction}
\label{sec:introduction}

Estimating the proportion of sites occupied by a species is common
challange for many subdisciplines ecology and evolution including
metapopulation, engangered and invasion biology. Greater acceptance of
the biases of imperfect detection has lead to the development and
proliferation of occupancy models, which model the occurrence of a
species at a site as a latent state layered underneath a detection
process \citep[e.g.,][]{mackenzie-2006, royle-2007-1813}. Now only a
little over a decade after occupancy models were introduced to
ecology, they are being used to model the occurence of everything from
bees \citep{mgonigle-2015-x} to tigers \citep{hines2010tigers} in an
endless variety of complexity.

Occupancy models are part of a larger class of models known as Hidden
Markov Models. For discrete Hidden Markov Models like occupancy models
where a species is either present or absent from a site, likelihood
calculation involves summing over the distribution of latent
states. Because estimating the effect of explanatory variables on site
occupancy or shared variation of in occupancy across species is often
of greatest interest to ecologsits
\citep[e.g.,][]{iknayan2014detecting}, the Hidden Markov Models are
embedded in a larger hierarchical model. In such cases, practitioners
may rely on Markov chain Monte Carlo (MCMC) to perform a Bayesian
analysis. Such models are computationally intensive, and large models
requiring hundreds or thousands of dimensions which require MCMC can
be computationally impractical. To ensure occupancy models are
accessible to practitioners, more efficient methods for estimating these
models are necessary. 

%% model selection

\section*{Materials \& Methods}
\label{sec:methods}


\section*{Results}
\label{sec:results}

\section*{Discussion}
\label{sec:discussion}

\section*{Acknowledgments}
\label{sec:acknowledge}

\bibliographystyle{mee}

\bibliography{refs}

% \begin{figure}
%   \centering
%   \includegraphics[width=.8\textwidth]{../analysis/changePoint/plotting/networks.pdf}
%   \caption{Assembling hedgerow networks had more changing points
%     (vertical red lines) than non-assembling hedgerows and weedy field
%     margins (a representative sample of non-assembling sites are
%     depicted here). In each network, plants and pollinators are
%     represented by green and yellow circles, respectively, weighted by
%     their degree. Each species has a consistent position in the
%     network across years. In the assembling hedgerows, colored squares
%     in the corner of each network represent the year of
%     assembly. Before hedgerows are installed (when they are still
%     field margins) the year of assembly is zero.}
%   \label{fig:changePoints}
% \end{figure}
% \clearpage

\end{document}

%%% Local Variables:
%%% mode: latex
%%% TeX-PDF-mode: t
%%% End:
